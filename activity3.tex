\subsection{이렇게 해보자.}

\subsubsection{static 메서드로 startActivity를 실행한다.}
\begin{lstlisting}[frame=single]
public class TodoDetailActivity extends FragmentActivity {
	
	private static final String TODO_ID = "todoId";
	private static final String TODO_CONTENT = "content";
	
	private static long todoId = 0L;

	public static void startActivity(Context context, long todoId, String content) {
		Intent intent = new Intent(context, TodoDetailActivity.class);
		intent.putExtra(TODO_ID, todoId);
		intent.putExtra(TODO_CONTENT, content);
		context.startActivity(intent);
	}
\end{lstlisting}
이것은 개발자 가이드에서 Fragment에 Argument를 전달할때 쓰는 방식인데, Activity에서도 활용할만 하다.\\
Intent에 key-value를 전달할 일이 많고, key는 문자열인데, 이 문자열을 Caller와 Callee 양쪽에서 직접 문자열을 넣는 방식은 오타 가능성이 있기 때문에 잘 쓰지 않는다.\\
key를 상수로 갖고 있는 게 좋은데 Caller 쪽에 있는 게 좋을지, Callee 쪽에 있는 게 좋을 지 또 망설여진다. 자주 쓰이는 key라면 별도의 클래스에 상수로 두는 방법도 있겠지만..\\

Callee 쪽에 있다고 한다면, 아래 코드 처럼 된다. 
\begin{lstlisting}[frame=single]
Intent intent = new Intent(this, TodoDetailActivity.class);
intent.putExtra(TodoDetailActivity.TODO_ID, todoId);
intent.putExtra(TodoDetailActivity.TODO_CONTENT, content);
startActivity(intent);
\end{lstlisting}
뭔가 깔끔한 맛이 없다. 이럴 때 Callee 쪽에 static 메서드로 startActivity를 만들어 놓는다면, Caller 쪽에서는 key 값을 신경 쓸 필요가 없고, Callee 쪽에서만 상수를 만들고 이 상수에서 꺼내서 사용하면 된다.\\
전달하는 값이 정해져 있는 경우에 쓸 만한 방법으로 대부분이 이 경우에 해당한다.
그렇지 않을 때는 static  메서드를 너무 많이 만들 수도 없으므로 그럴 때는 Calee 쪽에 상수를 두고서 Calee 쪽에서 참조하는 방법을 그냥 쓰도록 하자.\\

\subsubsection{여러 위젯이 있을 때 동일한 위젯끼리 모아놓는다.}
\begin{lstlisting}[frame=single]
Button okButton, detailButton;
Spinner groupSpinner, statusSpinner;
\end{lstlisting}

\begin{lstlisting}[frame=single]
Button okButton;
Spinner groupSpinner;
Button detailButton;
Spinner statusSpinner;
\end{lstlisting}

Activity에 동작이 많으면 코드가 길어지기 마련이다. 
예를 들어 이벤트 처리시에 ``어떤 스피너를 바꿔야 하지?''할 때가 있는데, 아래 방식보다는 위 방식에서 찾기가 수월하다.\\
AdapterView에서 ViewHolder 패턴을 쓸때도 마찬가지로 동일한 위젯끼리 모아놓는 것이 좋다.
(물론 findViewById() 해서 멤버 변수에 대입하는 것은 레이아웃 순서에 맞게 한다.)\\
동일한 위젯이 많은 때는 Grouping을 다시 할 수도 있다.
\begin{lstlisting}[frame=single]
Button mondayButton, tuesdayButton, wednesdayButton, thursdayButton, fridayButton, saturdayButton, sudayButton;
Button okButton, cancelButton;
\end{lstlisting}

\subsubsection{상수나 멤버 변수는 쓰이는 위치 근처에 선언한다}
무조건 맨 위에다가 멤버 변수를 모두 선언할 필요는 없다. 
메서드 한두개에서만 쓰는 멤버 변수는 바로 그 위에다가 선언해서 변수가 어디에 있고 어디에 쓰이는지 확인하는데 시간을 들이지 않도록 하자.

\subsection{이런거 하지 말자.}

\subsubsection{불필요한 Casting은 하지 말자.}
View나 ViewGroup에 Actvity상에서 기껏해야 Visibility만 조건에 따라 변경되는 경우가 있다. 이럴 때 굳이 findViewById()해서 LinearLayout이니 ImageView니 Casting하는 수고는 하지 말자.\\
% 샘플?
그리고 특히 Layout의 경우에는 변경될 가능성이 아주 많다. 버튼 두개면 LinearLayout 이면 충분한데, 그 버튼 한쪽에 New 마크가 붙는 사소한 이유로 RelativeLayout으로 바뀌어야 할 수가 있다. Layout param 변경이 필요한 경우에만 Casting을 한다.

\subsubsection{상속 구조를 깊게 하지 말자.}
복잡한 앱에서는 액티비티의 상속구조가 4, 5단계까지 있는 것을 본 적도 있다. 이런 코드를 보면, 분석이 상당히 어렵고, 문제가 발생할 때 찾아내는 것에 어려움이 많다.
단순한 유틸리티 메서드를 위해서 상속을 하지는 말자. 로직의 흐름상 반드시 수행해야 할 코드를 중심으로 상위 클래스를 만든다.\\

물론 쉽지는 않다. 클래스 구성을 잘 단순화 해야 하고, 여러 수단을 동원해야만 한다. 내 경우에는 AOP나 Marker Interface 도입 등을 통해서 클래스 구성을 단순화 한 적이 있다. 이 부분에 대해서는 다른 절에서 상세하게 다루기로 하자.\\

\subsubsection{Activity에서 너무 많은 Listener 인터페이스를 구현하지 말자}
OnTouchListener, OnClickListener, OnItemSelectedListener 등 안드로이드에서 제공하는 Listener 인터페이스가 많고, 개발중에도 필요에 의해서 Listener 인터페이스를 만들게 된다.\\

이런 인터페이스들을 Activity 선언에 모두 implements로 해놓고 setOnTouchListener(this), setOnClickListner(this), setOnItemSelectdLister(this) 이런 식으로 하고, 커스텀으로 만든 것까지 하면 여러 개가 될 수가 있다. 이렇게 하면 해당 이벤트의 Listener를 수정할 때, 메서드를 찾아가는 게 간단치 않다. OnClickListener 같은 경우는, onClick 메서드를 찾으면 되지만, OnItemSelectedListener 같은 건 어떨까? Listener에 메서드가 2개이다. 커스텀으로 만든 경우에는 여러 메서드를 가질 수도 있다.\\ 
내가 원하는 메서드를 금방 찾을 수 있을까?\\

이럴 때는 멤버 변수로 Listner 구현을 두는 것으로 전환하는 것도 생각해보자.

\begin{lstlisting}[frame=single]
private OnItemSelectedListener onItemSelectedListener = new OnItemSelectedListener() {

	@Override
	public void onItemSelected(AdapterView<?> parent, View view, int position, long id) {
		....
	}
	
	@Overrides
	public void onNothingSelected(AdapterView<?> parent) {
		...
	}

}	
\end{lstlisting}	
이렇게 하면 setOnItemSelectdLister(this)가 아닌 setOnItemSelectdLister(onItemSelectedListener)를 해놓으면 찾아가는 게 그 전보다 쉬워진다. OnItemSelectedListener의 onNothingSelected()처럼 메서드가 어디에 속한 것인지 익숙하지 않은 것들은 Activity의 메서드로 나와있는 것보다는 위처럼 하는 것을 권장한다.\\

또한 Button 같은 것이 위에서 얘기한 것처럼 Grouping이 되어 있다고 할 때 각각 OnClickListener를 만들어놓는 것이 낫다.
예를 들어 알람설정 화면에 월/화/수/목/금/토/일 같은 요일을 선택하는 Button이 있고, 알람/진동 같은 옵션도 Button으로 있다고 할 때, 같은 OnClickListener를 쓰면 코드를 깔끔하게 만들기 어렵다. 버튼 선택시 다중 선택이  되는 경우도 있고, 하나를 선택하면 다른 것의 선택이 해제되는 동작들 같은 것도 볼 수 있는데, if 문 안에서 번거롭게 처리하는 것보다 그룹별로 따로 Listener를 분리해놓고 생각하자.\\

아래 코드는 5분, 10분, 30분 3개의 버튼 가운데서 하나만 선택되어야 하는 것인데, 다른 Listener하고 분리해서 사용한 예이다.
\begin{lstlisting}[frame=single]
	@Override
	protected void onCreate(Bundle savedInstanceState) {
		...
		mSnoozeFiveMin.setOnClickListener(snoozeListener);
		mSnoozeTenMin.setOnClickListener(snoozeListener);
		mSnoozeThirtyMin.setOnClickListener(snoozeListener);
	}

	private OnClickListener snoozeListener = new OnClickListener() {
		
		@Override
		public void onClick(View view) {
			mSnoozeFiveMin.setSelected(view == mSnoozeFiveMin);
			mSnoozeTenMin.setSelected(view == mSnoozeTenMin);
			mSnoozeThirtyMin.setSelected(view == mSnoozeThirtyMin);
		}
		
	};
\end{lstlisting}	

\begin{comment}
\subsubsection{스레드에서 startActivity를 하지 말자.}
스레드에서 UI 코드가 동작 하지 않는다고 하지만, 그것은 어디까지나 ViewRootImpl의 checkThread를 거치는 경우 뿐이다.(즉 draw/redraw 하는 경우) 스레드에서 startActivity해도 동작은 한다.
\end{comment}