\chapter{BroadcastReceiver}
BroadcastReceiver는 Observer 패턴\footnote{\url{http://warmz.tistory.com/751}를 참고하자.}을 안드로이드에서 구현한 방식이다.
한 쪽에서 뭔가 특별한 이벤트가 발생했을때 다른 쪽에서도 해야 할 작업이 있다.
예를 들어, 안드로이드 책이 한 권 출판되어서(1) 출판사에서 연락하면(2), 각 온라인 서점에서는 데이터를 각각 입력해야만 한다(3). BroadcastReceiver 입장에서 보면 1번은 Intent 내용, 2번은 sendBroadcast() 메서드 호출, 3번은 onReceive() 메서드에 대응된다.\\

Observer 패턴에서는 일대다 관계에서 직접 호출하지 않고, 인터페이스를 통한 느슨한 결합을 통해 Observer를 register/unregister하는 방법을 제공한다.
이 방식은 BroadcastReceiver에서도 마찬가지이다. BroadcastReceiver는 바로 Observer이고, 이벤트는 sendBroadcast()에 전달되는 Intent이다(이벤트는 여러 곳에 퍼뜨리기 위한 방송용 이벤트라고 보면 된다). Context에는 registerReceiver()와 unregisterReceiver() 메서드가 있어서 여러 컴포넌트(Activity, Service, Application)에서 사용할 수 있다.

\section{BroadcastReceiver 구현}
BroacastReceiver에는 추상 메서드가 하나뿐이고 나머지 메서드는 사용할 일이 많지 않다.
\begin{verbatim}
	abstract void onReceive(Context context, Intent intent)
\end{verbatim}

BroadcastReceiver는 ContentProvider와 마찬가지로 ContextWrapper 하위 클래스가 아니다. 그러나 Context는 전달되므로 이것을 가지고 할 수 있는 게 있다.\footnote{onReceive() 메서드에 전달되는 Context는 Application의 Context이다.}
startService(), startActivity() 외에도 sendBroadcast()를 다시 호출할 수도 있다. 
이벤트가 발생하면 그 이벤트에 대응해서 화면을 보여주거나 백그라운드 작업이 필요한 경우가 많기 때문에, 실제로 실행하는 주된 일이 startService()나 startActivity()이다. \\

방송용 이벤트가 발생할 때 직접적으로 startService()나 startActivity()를 실행하는 방법은 없다. 이벤트는 바로 방송의 형태를 띄기 위해서 sendBroadcast()를 통해서 전달되고, 이때 화면을 띄우려면 BroadcastReceiver의 onReceive() 메서드에서 startActivity()를 실행하고, UI가 없는 내부 작업이 필요하다면 startService()를 실행한다.\\

Line이나 카카오톡 같은 메신저 앱의 경우를 보자. 누군가 내게 메시지를 보내면 Push Message를 BroadcastReceiver에서 받는다. 그 순간에 `띵동' 소리가 나고 화면에 `팟' 하면서 팝업에 대화 내용이 뜨는데, 이 팝업이 바로 onReceive() 메서드에서 startActivity()를 실행한 결과이다.\\

onReceive()에서 startService()와 startActivity()를 둘 다 실행하는 경우도 있다. 특정 이벤트가 발생할 때 배경 음악(화면이 닫혀도 배경 음악이 유지되어야 한다고 가정)이 깔리면서 화면이 떠야 한다면, 배경 음악이 깔리는 것은 startService()를 실행해야 하고 화면이 뜨는 것은 startActivity()를 실행해야 한다.\\

onReceive() 메서드는 메인 스레드에서 실행되므로 시간 제한도 있다.
10초(포그라운드)/1분(백그라운드: 기본)내에 onReceive() 메서드 실행이 끝나야 한다. 10초/1분이 넘으면 ANR이 발생한다.
메인 쓰레드에서 실행되므로 실제로는 1분/10초/5초도 아닌 훨씬 짧은 시간내에 처리가 완료되어야 한다. 그렇지 않으면 UI에서 버벅거림의 원인이 될 수 있다.\\

한 이벤트에 대해서 한 앱에 여러 BroadcastReceiver가 등록된 경우도 주의해야 한다.
본래 Broadcast는 여러 프로세스 간에 이벤트를 전달하기 위한 것이다. 각 프로세스라면 동시에 실행되는 데 문제가  없지만, 한 앱에서는 여러 개의 BroadcastReceiver가 있을 때 onReceive() 메서드는 순차적으로 하나씩 실행되기 때문에 BroadcastRecever를 실행하느라 UI 동작에 문제가 생길 수 있다.\\

\begin{comment}
어떤 앱에서는 Application에만 registerReceiver를 해놓는 경우도 있다. BroadcastReceiver를 최소한의 갯수로 하고, 최소한의 시간만 존재하도록 하는 것이 BroadcastReciever로 인해 UI가 버벅거리지 않게 하는 방법이다.
\end{comment}

\section{BroadcastReceiver 등록}
BroadcastReceiver를 등록하는 방식은 정적인 등록(statically publish)과 동적인 등록(dynamically register) 2가지가 있다.\footnote{등록이라고 번역하면서 영어로는 publish와 register로 구분해서 쓰고 있다. 정적인 등록은 공개의 의미가 있다고 이해하자. 앱을 설치하자마자 곧바로 BroadcastReciever가 사용 가능하다.}\\

정적인 등록은 AndroidMainfest.xml에 추가해서 해당 이벤트에 항상 반응하는 것이다. 주로 시스템 이벤트를 받을 때 많이 사용하는 것으로 앱 실행 중이 아닐 때에도 이벤트에 대한 처리를 하게 된다.
아래는 ApiDemos에 등록된 BroadcastReceiver의 예이다.
\begin{lstlisting}[frame=single] 
<receiver android:name=".os.SmsMessageReceiver">
	<intent-filter>
		<action android:name="android.provider.Telephony.SMS_RECEIVED" />
	</intent-filter>
</receiver>
\end{lstlisting}

Intent API 문서에 보면 시스템의 Broadcast Action 상수가 다수 정의돼 있다.\footnote{ACTION\_DATE\_CHANGED Action은 버그가 있다. 자정에만 오면 되는데 정오에도 전달되는 문제가 있다.
%이후에 AlarmManagerService를 얘기할 때 다시 살펴보기로 하자.
} API 문서에서 ``Broadcast Action:'' 으로 검색해보자.

\begin{verbatim}
ACTION_TIME_TICK
ACTION_TIME_CHANGED
ACTION_TIMEZONE_CHANGED
ACTION_DATE_CHANGED
ACTION_BOOT_COMPLETED
ACTION_PACKAGE_ADDED
ACTION_PACKAGE_CHANGED
ACTION_PACKAGE_REMOVED
ACTION_PACKAGE_RESTARTED
ACTION_PACKAGE_DATA_CLEARED
ACTION_UID_REMOVED
ACTION_BATTERY_CHANGED
ACTION_BATTERY_LOW
ACTION_BATTERY_OKAY
ACTION_POWER_CONNECTED
ACTION_POWER_DISCONNECTED
ACTION_SHUTDOWN
ACTION_AIRPLANE_MODE_CHANGED
ACTION_CAMERA_BUTTON
ACTION_CONFIGURATION_CHANGED
ACTION_HEADSET_PLUG
ACTION_LOCALE_CHANGED
ACTION_MEDIA_SCANNER_FINISHED
ACTION_MEDIA_SCANNER_SCAN_FILE
ACTION_MEDIA_SCANNER_STARTED
ACTION_MY_PACKAGE_REPLACED
ACTION_SCREEN_OFF
ACTION_SCREEN_ON
\end{verbatim}

문서를 보면 `This is a protected intent that can only be sent by the system.' 메시지가 많이 나온다. 즉 해당 이벤트는 시스템이 발생시키는 것이지, 앱에서 발생시킬 수는 없는 것이다.
이런 protected intent를 발생시켜보면 SecurityException을 만나게 된다. 아래는 
sendBroadcast(new Intent(Intent.ACTION\_BOOT\_COMPL\-ETED))를 실행한 결과 예외 스택이다.
\begin{lstlisting}[frame=single] 
07-29 21:53:20.312: E/AndroidRuntime(15102): Caused by: java.lang.SecurityException: Permission Denial: not allowed to send broadcast android.intent.action.BOOT_COMPLETED from pid=15102, uid=10301
07-29 21:53:20.312: E/AndroidRuntime(15102): 	at android.os.Parcel.readException(Parcel.java:1465)
07-29 21:53:20.312: E/AndroidRuntime(15102): 	at android.os.Parcel.readException(Parcel.java:1419)
07-29 21:53:20.312: E/AndroidRuntime(15102): 	at android.app.ActivityManagerProxy.broadcastIntent(ActivityManagerNative.java:2430)
07-29 21:53:20.312: E/AndroidRuntime(15102): 	at android.app.ContextImpl.sendBroadcast(ContextImpl.java:1241)
07-29 21:53:20.312: E/AndroidRuntime(15102): 	at android.content.ContextWrapper.sendBroadcast(ContextWrapper.java:365)
\end{lstlisting}

Intent 클래스에 정의된 모든 Broadcast Action이 `protected intent'인 것은 아니다.
ACTION\_MEDIA\_SCA\-NNER\_SC\-AN\_FI\-LE Action 같은 것은 앱에서 발생시키라고 있는 것이다.
이미지를 SD Card에 저장했는데 미디어 스캐닝이 안 돼서 곧바로 화면에 가져올 수 없는 경우가 있다.
이때 바로 앱에서 사용하는 방식이 ACTION\_MEDIA\_SCANNER\_SCAN\_FILE Action를 브로드캐스팅해서 미디어 스캐너가 동작하게 하는 것이다.\\

캘린더 앱이라면, 처리해야 하는 이벤트로는 ACTION\_TIMEZONE\_CHANGED,
ACTION\_LOCALE\_C\-HANG\-ED Action이 있다.
그리고 많은 앱에서 처리하는 이벤트로는 ACTION\_BOOT\_COMPLETED Action도 있다. 
앱에서 특정 시간에 하는 작업을 위해서는 AlarmManager에 등록하고 있다.
PendingIntent를 setXxx() 메서드를 통해서 특정 시간에 동작하도록 AlarmManager에 등록하는데, 이 알람들은 단말이 꺼지면 모두 사라져 버리므로 다시 부팅하면 사라진 알람들을 재등록해야 한다. 이때 ACTION\_BOOT\_COMPLETED Action을 받아서 재등록을 진행한다.\\

동적인 등록은 registerReceiver()/unregisterReceiver() 메서드로 등록과 해제를 한다.
일반적으로 Activity에서는 포그라운드 lifetime인 onResume()/onPause()에서 registerReceiver()/unregisterReceiver()를 호출한다.
Activity가 visible lifetime이나 entire lifetime에서 이벤트를 받아야 하는 경우도 물론 있다.
등록할 때는 먼저 IntentFilter를 생성해서 Action을 추가하고 BroadcastReceiver를 만든다. 그리고 IntentFilter와 BroadcastReceiver를 registerReceiver()에 파라미터로 전달한다.
\begin{lstlisting}[frame=single] 
	private static final String VOLUME_CHANGED_ACTION = "android.media.VOLUME_CHANGED_ACTION";

	private BroadcastReceiver receiver = new BroadcastReceiver() {
		
		@Override
		public void onReceive(Context context, Intent intent) {
			if (intent.getAction().equals(VOLUME_CHANGED_ACTION)) {
				int value = intent.getIntExtra("android.media.EXTRA_VOLUME_STREAM_VALUE", -1);
				if (value > -1) {
					mVolumeSeekBar.setProgress(value);
				}
			}
		}
	};
	

	@Override
	protected void onResume() {
		super.onResume();
		IntentFilter filter = new IntentFilter();
		filter.addAction(VOLUME_CHANGED_ACTION);
		registerReceiver(receiver, filter);
	}
	
	@Override
	protected void onPause() {
		unregisterReceiver(receiver);
		super.onPause();
	}	
\end{lstlisting}

이 코드는 Volume 키를 통해서 소리 크기가 바뀌면, 화면의 SeekBar도 그에 맞게 바뀌는 내용이다. 
그런데 VOLUME\_CHANGED\_ACTION Action은 안드로이드 API 문서에 없는 내용이다.
코드 상으로는 AudioManager 클래스에 상수로 있는데 @hide로 숨겨져 있다. 
이런 식으로 Action명을 알 수 있다면 처리할 수 있는 케이스가 많다.\\

바탕화면에 Shortcut을 설치하는 것도 Broadcast를 통해서 한다. com.android.launcher.action.INSTALL\_SH\-ORT\-CUT도 undocumented Action이다.
\begin{lstlisting}[frame=single] 
	Intent shortcutIntent = new Intent("android.intent.action.MAIN", null);
		
	PackageManager pm = context.getPackageManager();
	Intent launchIntent = pm.getLaunchIntentForPackage(context.getPackageName());
	String className = launchIntent.getComponent().getClassName();
	shortcutIntent.setClassName(context, className);
	shortcutIntent.addFlags(Intent.FLAG_ACTIVITY_NEW_TASK | Intent.FLAG_ACTIVITY_RESET_TASK_IF_NEEDED);
	shortcutIntent.addCategory(Intent.CATEGORY_LAUNCHER);

	Intent intent = new Intent();
	intent.putExtra(Intent.EXTRA_SHORTCUT_INTENT, shortcutIntent);
	intent.putExtra(Intent.EXTRA_SHORTCUT_NAME, context.getString(R.string.app_name));
	intent.putExtra(Intent.EXTRA_SHORTCUT_ICON_RESOURCE, Intent.ShortcutIconResource.fromContext(context, R.drawable.icon));
	intent.setAction("com.android.launcher.action.INSTALL_SHORTCUT");
	intent.putExtra("duplicate", false);
	context.sendBroadcast(intent);
\end{lstlisting}
여기에 필요한 퍼미션은 문서에 나와 있다.\\
\url{http://developer.android.com/reference/android/Manifest.permission.html}
\begin{verbatim}
<uses-permission android:name="com.android.launcher.permission.INSTALL_SHORTCUT" />
\end{verbatim}

안드로이드 소스를 받아보면 Launcher2라는 론처에서 해당 퍼미션을 정의하고, 해당 Action에 대한 BroadcastReceiver가 있다.
Launcher2도 우리가 만드는 앱과 동등한 레벨인 홈 애플리케이션일 뿐이지만, 단말에 반드시 있는 앱(홈 애플리케이션/캘린더/주소록)에 대해서는 Manifest.permission.html 문서에 퍼미션을 정의하고 있는 것을 볼 수 있다.\\

% intent-filter에 정의되지 않은 것도 받는 것인지 체크

\section{Ordered Broadcast}
sendOrderedBroadcast(Intent intent, String receiverPermission) 메서드는 등록된 BroadcastReceiver 가운데 priority가 높은 순으로 전달한다.\footnote{파라미터에 Intent만 전달되는 메서드는 없으며, 권한이 불필요하면 null로 전달한다. sendOrderedBroadcast()는 파라미터 개수가 다른 두 개의 메서드가 더 있다.}
priority는 AndroidManifest.xml에서 intent-filter에 android:priority를 쓸 수도 있고, IntentFilter에 setPriority(int priority) 메서드로 정할 수도 있다. priority가 동일하면 실행 순서는 랜덤이다.\\

Ordered Broadcast는 여러 BroadcastReceiver를 순서대로 진행하면서  BroadcastReceiver 간에 결과를 넘겨줄 수도 있고, Broadcast를 도중에 중지시킬 수도 있다. Broadcast가 가벼운 작업이 아닌데 결과를 넘겨가면서 계속해서 진행한다는 의미보다는, 결과를 넘겨가면서 적정 시점이 되면 중지시키기 위함이라고 보는 게 더 적절하다.\\

예를 들어보자. 화면이 여러 개 떠 있는 상태에서, 백그라운드 Service에서 서버와 동기화 작업을 하다가 인증 문제(AUTH\_FAILED)가 발생했다. 이때 원하는 것은 ``인증 실패 발생'' 메시지를 화면에 보여주고 `` 확인'' 버튼을 클릭하면 떠있는 화면들을 모두 종료하는 것이다.
Service에서도 Activity를 띄울 수는 있는데, 인증 실패 메시지 화면 외에 다른 떠 있는 화면들을 종료하는 것은 어떻게 할까?\\

이때 가장 먼저 떠오르는 방법이 AUTH\_FAILED Action를 받는 BroadcastReceiver를 Activity마다 등록해서 onReceive()에서 Activity의 finish() 메서드를 호출하는 것이다.
상위 클래스인 BaseActivity의 onCreate()에서 registerReceiver()로 등록하는 것이 가능하지만, 하나씩 finish()를 실행하면 뭔가 비효율적이다. 젤리빈부터 가능하긴 하지만, finishAffinity() 메서드로 Ordered Broadcast의 활용 방법을 생각해보자.\\

finishAffinity() 메서드는 Activity 어느 하나에서만 호출하면 되는데, registerReceiver()를 실행할 때 IntentFilter에 setPriority()로 우선순위를 지정하고, 가장 먼저 onReceive()를 실행하는 곳에서 finishAffinity()를 호출하고서 Broadcast를 중지(abort)시킬 수 있다.

\begin{lstlisting}[frame=single] 
	private BroadcastReceiver receiver = new BroadcastReceiver() {
		
		@Override
		public void onReceive(Context context, Intent intent) {
			finishAffinity();
			abortBroadcast();
		}
	};
	

	@Override
	protected void onCreate() {
		super.onCreate();
		IntentFilter filter = new IntentFilter();
		filter.addAction(CalendarIntent.AUTH_FAILED);
		registerReceiver(receiver, filter);
		....
	}
\end{lstlisting}
이런 식으로 어디서든 한쪽에서만 실행하고 끝나도 되는 것이라면, Ordered Broadcast를 고려해보자.\\

\section{Sticky Broadcast}
sendStickyBroadcast()는 Intent를 시스템에 등록해놓고, 새로 해당 Intent를 받을 수 있는 BroadcastReceiver가 등록되면 이 시점에 BroadcastReceiver의 onReceive()가 호출된다. 즉 먼저 이벤트를 발생시켜 놓고서, 그 이벤트 상태를 알고자 하는 BroadcastReceiver가 등록되는 시점에 BroadcastReceiver에서 이벤트를 받는 것이다.
Sticky Broadcast를 보내기 위해서는 android.permission.BROADCAST\_STICKY 퍼미션이 필요하다.\\

시스템에서 보내는 Sticky Broadcast는 Intent API 문서에서 확인할 수 있다. ACTION\_BATTERY\_CHANGED, ACTION\_DEVICE\_STORAGE\_LOW, ACTION\_DOCK\_EVENT Action에 대해서 BroadcastReceiver를 등록해서 배터리 상태나 저장소 부족 여부, 그리고 도킹 상태를 알 수 있다.\\

앱에서는 sendStickyBroadcast()를 호출하는 것은 권장되지 않는다. 롤리팝에서는 이 메서드가 deprecated 되기도 했다.
Sticky Broadcast는 계속 시스템 메모리에 정보가 남아있어서, 어디선가 정보를 알고 싶을 때 언제든지 빼낼 수가 있기 때문에 보안 문제가 있다.
이를테면 여러 앱 간에 SSO(Single Sign-On) 기능을 구현하고자 한다. 
한 쪽에서 로그인하면 해당 아이디로 다른 앱에서도 로그인 되고, 명시적으로 로그아웃하면 다른 쪽에서도 로그아웃되는 게 SSO의 기본 기능이다. 이때 Sticky Broadcast를 쓰고 싶은 유혹이 있다. 
Sticky Broadcast를 사용해서 Intent에 로그인 여부와 로그인 아이디 등을 전달하면 어떨까? 
onResume()에서 registerReceiver()를 실행한다면, 앱이 포그라운드로 올 때마다 최신 정보를 알 수 있고 그에 맞게 처리할 수 있다. 그런데 이 정보는 다른 앱에서도 빼낼 수 있고, Intent 정보를 다른 것으로 바꿔치기 하고서 sendStickyBroadcast()를 다시 할 수도 있다.
쓰기에 따라서 유용하지만 중요한 정보에 대해서 sendStickyBroadcast()를 쓰지 않는 것이 좋다. 
% http://blog.naver.com/PostView.nhn?blogId=huewu&logNo=110090854068
% 사용 예가 필요하다. 

\begin{comment}
%\subsubsection{Async 작업}
onReceive 메서드가 끝나면, BroadcastReceiver가 끝나서, 스레드로 작업 할 수는 없으므로, 
일반적으로 시간이 오래 걸리는 작업이 있다면, BroadcastReceiver에서 Service를 실행시킨다.

내부적으로 unregister를 하고 있다.
LoadedApk?
07-01 10:28:19.472: E/ActivityThread(4568): Activity com.suribada.misc.BroadcastRecieverActivity has leaked IntentReceiver \verb|com.suribada.misc.BroadcastRecieverActivity$1@b5a4f558 that was originally registered here. Are you missing a call to unregisterReceiver()?|
\end{comment}

\section{LocalBroadcastManager}
Context의 sendBroadcast() 메서드는 Binder IPC를 통해 ActivityManagerService를 거쳐야 하므로 속도에서 이점이 크지 않다. 또한 Intent Action을 안다면 원치 않는 곳에서도 이벤트를 받아서 예기치 않는 일을 할 가능성도 있다.
sendBroadcast()에서는 원하는 패키지만 전달받을 수 있도록 하는 방법을 제공하는가?
바로 Intent의 setPackage() 메서드로 Broadcast 범위를 줄일 수 있다. 
다만 이 방식은 ICS부터 동작한다.\\

프로세스 간에 Broadcast를 보낼 필요가 없다면 support-v4에 포함된 LocalBroadcastManager 사용을 고려하자. 
LocalBroadcastManager는 로컬 프로세스에서만 이벤트를 주고받을 수가 있다.
시스템 Broadcast와는 구조가 완전히 다른데 ActivityManagerService를 거치지 않고 싱글톤인 LocalBroadcastManager에서 BroadcastReceiver를 등록하고 sendBroadcast()를 실행한다.

\begin{lstlisting}[frame=single] 
LocalBroadcastManager.getInstance(this)
	.sendBroadcast(CalendarIntent.CHANGE_TIME);
...
LocalBroadcastManager.getInstance(this).registerReceiver(..)
\end{lstlisting}

LocalBroadcastManager가 좋은 점은 일반적으로 두 가지를 얘기한다.
\begin{itemize}
\item Broadcast하는 데이타가 다른 앱에서 캐치되지 않아서 더 안전하다.
\item 글로벌 Broadcast보다 속도가 빠르다.
\end{itemize}

LocalBroadcastManager의 sendBroadcast() 메서드는 내부적으로 등록된 BroadcastReceiver를 바로 실행하지 않고 Handler에 처리하라고 메시지를 보내서 다음에 메인 스레드가 가능한 시점에 처리한다. 메인 Looper MessageQueue에 쌓여있는 게 많다면 처리가 늦을 수도 있다.\\

그런데 인증 에러같은 것은 MessageQueue에서 처리하면 문제가 될 수 있다.
BroadcastReceiver에서 바로 처리해야 하는데, 그렇지 않고 다른 작업을 하다가 타이밍상 엉뚱한 결과를 만들어 내는 경우가 생긴다. 
이럴 때 쓰는 것이 sendBroadcastSync() 메서드이다. 
등록된 BroadcastReceiver은 모두 그 순간에 처리하는데, 다만 이때는 sendBroadcastSync()와 onReceive()는 동일한 스레드에서 실행된다는 제약이 있다.\\

LocalBroadcast에서 주의할 점은 BroadcastReceiver의 한 종류인 App Widget에는 이벤트가 전달되지 않는다는 것이다. App Widget이 설치된 위치는 홈 스크린으로 별도 프로세스에 있으므로, 일반 Broadcast를 사용해서 이벤트를 전달해야만 한다. 다만 App Widget의 onReceive() 실행 위치는 다시 앱의 프로세스이다.

\begin{comment}
별도 스레드내에서 sendBroadcastSync를 할 수 밖에 없다고 하면 onReceive 안에다가 UI를 건드리는 쪽에는 runOnUIThread 해도 되지만 참 번거로운 일이 되네요.

Activity의 runOnUIThread는 호출하는 곳이 uiThread가 아니면 역시 post를 해주고 있어요.
runOnUIThread는 해결책이라고 볼 수가 없네요.

\end{comment}

\section{App Widget}
App Widget은 다른 애플리케이션에 내장되어서(embeded), 주기적으로 업데이트하는 작은 애플리케이션(miniature application views)이다. 데스크탑 바탕 화면이나 웹브라우저에서 일부 영역을 차지하고 있는 동그란 시계 위젯같은 것을 안드로이드에서도 제공한다고 보면 된다.\\

보통 Launcher의 홈 스크린에서 AppWidgetHost를 구현해서 App Widget을 내장한다. 젤리빈 API 레벨 17부터 락 스크린에서도 App Widget이 내장된다. 홈 스크린이나 락 스크린이 아닌 일반 앱 화면에서도 AppWidetHost를 구현할 수는 있지만 따라야 할 제약(contractual obligation)이 많아서 권장되지 않는다.

\subsection{App Widget 기본}
App Widget의 기본 특성을 알아보자.
\begin{enumerate}
\item 설치되는 위치(Launcher 프로세스)와 실행되는 위치(앱 프로세스)가 다르다. AppWidgetService(App Widget 목록, 이벤트 발생)가 실행되는 system\_server까지 포함하면 관련 프로세스는 모두 3개이다.
\item App Widget 변경은 Broadcast를 통한다. 주로 시스템(AppWidgetService)에서 sendBroadcast()를 호출하면 BroadcastReceiver의 onReceive() 메서드에서 App Widget에 작업을 하는 방식이다.
아래는 ApiDemos에 있는 App Widget 선언이다.
\begin{lstlisting}[frame=single] 
        <receiver android:name=".appwidget.ExampleAppWidgetProvider">
            <meta-data android:name="android.appwidget.provider"
                    android:resource="@xml/appwidget_provider" />
            <intent-filter>
                <action android:name="android.appwidget.action.APPWIDGET_UPDATE" />
            </intent-filter>
        </receiver>
\end{lstlisting}
여기서 중요한 것은 2라인에서 meta-data에 android.appwidget.provider가 전달된다는 것이다. 이를 통해 BroadcastReceiver는 App Widget 용도로 정해진다.
\end{enumerate}

BroadcastReceiver로도 App Widget을 만들 수는 있지만 대체로 AppWidgetProvider를 상속해서 App Widget을 만든다. 
AppWidgetProvider는 내부적으로 onReceive()에서 Intent Action으로 구분해서, onUpdate(), onDeleted(), onEnabled(), onDisabled() 메서드로 Intent extra 값을 전달한다.\\

AppWidgetProvider의 프레임워크 소스를 보자. 각 메서드가 언제 호출되는지와 각 메서드에 전달되는 appWidgetIds 등 파라미터의 의미를 알 필요가 있기 때문에 한번쯤은 보는 게 좋다.
\begin{lstlisting}[frame=single, caption=AppWidgetProvider.java] 
public class AppWidgetProvider extends BroadcastReceiver {

	public void onReceive(Context context, Intent intent) {
		String action = intent.getAction();
		if (AppWidgetManager.ACTION_APPWIDGET_UPDATE.equals(action)) { // (1)
			Bundle extras = intent.getExtras();
			if (extras != null) {
				int[] appWidgetIds = extras.getIntArray(AppWidgetManager.EXTRA_APPWIDGET_IDS);
				if (appWidgetIds != null && appWidgetIds.length > 0) {
					this.onUpdate(context, AppWidgetManager.getInstance(context), appWidgetIds);
				}
			}
		} else if (AppWidgetManager.ACTION_APPWIDGET_DELETED.equals(action)) { // (2)
			Bundle extras = intent.getExtras();
			if (extras != null && extras.containsKey(AppWidgetManager.EXTRA_APPWIDGET_ID)) {
				final int appWidgetId = extras.getInt(AppWidgetManager.EXTRA_APPWIDGET_ID);
				this.onDeleted(context, new int[] { appWidgetId });
			}
		} else if (AppWidgetManager.ACTION_APPWIDGET_OPTIONS_CHANGED.equals(action)) { // (3)
			Bundle extras = intent.getExtras();
			if (extras != null && extras.containsKey(AppWidgetManager.EXTRA_APPWIDGET_ID)
					&& extras.containsKey(AppWidgetManager.EXTRA_APPWIDGET_OPTIONS)) {
				int appWidgetId = extras.getInt(AppWidgetManager.EXTRA_APPWIDGET_ID);
				Bundle widgetExtras = extras.getBundle(AppWidgetManager.EXTRA_APPWIDGET_OPTIONS);
				this.onAppWidgetOptionsChanged(context, AppWidgetManager.getInstance(context), appWidgetId, 
						widgetExtras);
			}
		} else if (AppWidgetManager.ACTION_APPWIDGET_ENABLED.equals(action)) { // (4)
			this.onEnabled(context);
		} else if (AppWidgetManager.ACTION_APPWIDGET_DISABLED.equals(action)) { // (5)
			this.onDisabled(context);
		}
	}

	public void onUpdate(Context context, AppWidgetManager appWidgetManager, int[] appWidgetIds) {
	}
	
	public void onAppWidgetOptionsChanged(Context context, AppWidgetManager appWidgetManager, int appWidgetId,
			Bundle newOptions) {
	}

	public void onDeleted(Context context, int[] appWidgetIds) {
	}

	public void onEnabled(Context context) {
	}

	public void onDisabled(Context context) {
	}
	
}
\end{lstlisting}
\begin{itemize}
\item appWidgetId는 App Widget이 홈 스크린에 꺼내질 때마다 새로 받는 인스턴스 id 값이다. 홈 스크린에 설치되는 여러 App Widget은 모두 다른 인스턴스 id 값을 가진다. 즉 App Widget별로 있는 게 아닌 전역적인 값이다.
일반적으로 App Widget은 하나씩 제거되므로, ACTION\_APPWIDGET\_DELETED Action에는 하나의 appWidgetId만 전달된다.
\item 5라인(1)에서 ACTION\_APPWIDGET\_UPDATE Action은 부팅 시, 최초 설치 시, update interval 경과 시에 호출된다. App Widget에서 가장 주요한 Intent Action이다.
\item 13라인(2)에서 ACTION\_APPWIDGET\_DELETED Action은 인스턴스 하나가 삭제될 때 호출된다.
\item 19라인(3)에서 ACTION\_APPWIDGET\_OPTIONS\_CHANGED Action은 App Widget이 새로운 사이즈로 레이아웃될때 호출된다.(젤리빈부터)
\item 28라인(4)에서 ACTION\_APPWIDGET\_ENABLED Action은 최초 설치시에 호출된다.
\item 30라인(5)에서 ACTION\_APPWIDGET\_DISABLED Action은 마지막 인스턴스가 삭제될 때 호출된다.
\item ACTION\_APPWIDGET\_DELETED, ACTION\_OPTIONS\_CHANGED, ACTION\_ENABLED,\\
ACTION\_DISABLED Action은 protected intent로 시스템에서만 보낼 수 있다.
앱에서는 ACTION\_A\-PP\-WIDGET\_UPDATE만 보낼 수 있다.
\end{itemize}

AppWidgetProvider를 상속하면 onReceive() 메서드는 구현하지 않아도 되지만, 별도 Intent Action을 사용한다면 onReceive() 메서드를 오버라이드해야 한다. App Widget에 버튼 클릭 같은 액션이 있어서 다시 그려야 하는 경우 등이다.\\

\subsection{RemoteViews}
RemoteViews는 다른 프로세스에 있는 뷰 계층을 나타낸다. 주로 다른 프로세스에서 보여져야 하는 Notification이나 App Widget에서 사용하고 있다.
RemoteViews는 언뜻 보기에 일종의 ViewGroup으로 생각할 수 있는데, android.widget 패키지 안에 있을 뿐 View나 ViewGroup를 상속한 것이 아닌 Parcelable과 LayoutInflator.Filter 인터페이스를 구현한 일반 클래스일 뿐이다.\\

\colorbox{tearose}{\parbox[t]{15cm}{
Toast도 내부적으로 Notification과 동일하게 system\_server 프로세스의 NotificationManagerService를 사용한다. 그런데 Toast는 커스텀 화면으로 만들 때 왜 RemoteViews를 쓰지 않을까? Toast는 Binder Callback을 전달해서 Callback을 받는 로컬에서 띄우는 것이기 때문에 RemoteViews는 필요하지 않다.
}}\newline\newline

RemoteViews의 레이아웃에서 쓸 수 있는 뷰 클래스가 한정되어 있는데, LayoutInflator.Filter 인터페이스의  onLoadClass() 메서드가 뷰 클래스를 제한하는 역할을 한다. LayoutInflator.Filter 구현체인 RemoteViews에서 boolean onLoadClass(Class clazz) 메서드를 보면 클래스 선언에 @RemoteView 애노테이션이 있는 것만 true를 리턴한다.
이렇게 제한되어 사용가능한 뷰 클래스는 아래와 같다.\\

FrameLayout, LinearLayout, RelativeLayout, GridLayout,\\
AnalogClock, Button, Chronometer, ImageButton, ImageView, ProgressBar, TextView, ViewFlipper,\\
ListView, GridView, StackView, AdapterViewFlipper(이 4개는 허니콤부터 지원)\\

이런 제한을 알아도 거치게 되는 시행착오들이 있다. 그런 내용을 알아 보자.
\begin{itemize}
\item 애노테이션은 상속된 클래스는 해당하지 않기 때문에 위 목록에 있는 것의 하위 클래스도 RemoteViews에는 쓸 수 없다.
\item 커스텀 뷰도 쓸 수 없다. 클래스 선언에 @RemoteView를 추가하면 될 것 같지만, 설치되는 프로세스인 Launcher에서 이 커스텀 뷰를 찾을 방법이 없어서 그렇다.\footnote{\url{http://yenliangl.blogspot.kr/2010\_05\_23\_archive.html}을 참고하자.}
\item 최상위 클래스인 android.view.View는 쓸 수 없다. 레이아웃에 내용을 구분하기 위한 단순 라인 구분자(line separator)를 만들 때는 View에 background color을 넣으면 됐지만, RemoteViews에서는 TextView 같은 지원되는 뷰 클래스로 대체해야 한다.(이 경우 레이아웃에서 Lint warning에 나온다. 이런 warning은 무시하자).
\end{itemize}

RemoteViews는 뷰 계층 상에 있는 레이아웃 리소스 id를 대상으로 작업(TextView라면 텍스트 변경, 사이즈 변경, 텍스트 색상 변경 등)을 전달한다.
내부적으로는 이런 작업(Action) 목록을 가지고 있고, AppWidgetHost에서 이런 작업 목록을 한꺼번에 실행하는 방식이다.\\

RemoteViews를 통해서 지원 가능한 뷰 클래스에서 쓸 수 있는 메서드도 제한적일 거 같은데 그렇지는 않다. 
RemoteViews API 문서를 보면 메서드 가운데서 두번째 파라미터에 methodName 문자열이 전달되는데, 리플렉션을 통해 세번째 파라미터에 methodName을 전달하게 된다.
setImageViewBitmap(), setImageViewUrl(), setTextColor(), setTextViewText() 같은 메서드는 모두 내부적으로 이 리플렉션 용도의 메서드를 다시 호출한다.\\

\subsection{App Widget 업데이트}
App Widget을 업데이트하는 주요 코드는 간단하다.
\begin{lstlisting}[frame=single] 
	RemoteViews views = new RemoteViews(context.getPackageName(), R.layout.appwidget_provider); // (1)
	views.setTextViewText(R.id.appwidget_text, text); // (2)
	....
	appWidgetManager.updateAppWidget(appWidgetId, views); // (3)
\end{lstlisting}
\begin{itemize}
\item 레이아웃 리소스를 전달해서 RemoteViews를 생성한다.(1)
\item RemoteViews에 뷰 리소스별로 Action을 지정한다.(2)
\item AppWidgetManager.updateAppWidget() 메서드를 호출한다.(3)
\end{itemize}

그렇다면 이런 루틴들은 꼭 BroadcastReceiver에서 실행되어야 할까? 그렇지 않다. AppWidgetManager.getInstance(Context context)를 통해 Context가 전달되는 어디서든 AppWidgetManager를 얻을 수 있기 때문에, Activity나 Service에서도 동일하게 사용할 수 있다.
게다가 백그라운드 스레드에서 루틴을 실행해도 문제가 없다. Service에서 백그라운드 스레드로 App Widget을 업데이트를 해도 되는 이유이다.\\

updateAppWidget()을 호출할 때 호출 스택을 한번 따라가보면 아래와 같다.
\begin{lstlisting}[frame=single] 
AppWidgetManager.updateAppWidget
	AppWidgetService.updateAppWidgetIds
		AppWidgetServiceImpl.updateAppWidgetIds
			AppWidgetServiceImpl.updateAppWidgetInstanceLocked
				IAppWidgetHost.updateAppWidget[callback in AppWidgetHost]
					AppWidgetHostView.updateAppWidget[via Handler]
\end{lstlisting}
Parcelable인 RemoteViews는 계속해서 전달되고, AppWidgetHostView에서 RemoteViews의 Action 목록을 한꺼번에 처리한다.

\subsection{유의할 점}
\subsubsection{메인 스레드 점유}
onReceive() 메서드는 당연히 메인 스레드에서 실행된다. 따라서 여기서도 실행 속도에 주의해야 한다.
포그라운드에서 앱을 사용하는 중에, App Widget을 업데이트하기 위해 onReceive()가 실행된다면 버벅거림의 원인이 될 수 있다.
그렇다면 onReceive() 메서드 내에서 네트워크 통신이 필요하거나\footnote{메인 스레드에서는 NetworkOnMainThreadException 때문에 네트워크 통신을 할 수 없다.} DB에서 가져올 데이터가 많아 처리 시간이 많이 예상되는 경우에는 어떻게 할 것인가? 이런 경우에도 App Widget Update 작업을 Service로 넘기고 백그라운드 스레드에서 처리해야 한다.\\

onReceive() 메서드에서 AsyncTask를 실행해서 App Widget을 업데이트하는 코드를 본 적이 있는데, BroadcastReceiver는 onReceive() 메서드가 리턴되면 프로세스가 제거될 수 있기 때문에 실행을 보장할 수 없다.
좀 더 상세히 얘기해보자. 예를 들어, 앱의 Activity가 포그라운드에 있다면 프로세스 우선 순위가 높아서 문제가 되지 않는다. 그런데 이렇게 다른 컴포넌트가 실행 중인 게 없고, Widget update interval 시간이 되었거나 특정 Broadcast에 반응해서 BroadcastReceiver가 단독으로 실행된다면 어떨까? 
onReceive() 메서드가 끝나자마자 프로세스 우선 순위가 낮아져서 다른 프로세스에 밀리면 프로세스가 종료될 가능성이 있다. AsyncTask 결과가 나올 때까지 프로세스가 기다릴 수 없는 것이다.\\

결론적으로 App Widget 업데이트는 간단한 것 외에는 Service에서 실행하는 것이 권장된다. 

\subsubsection{부팅 중에는 initialLayout만 보임}
캘린더 일정 목록 App Widget이 있다고 하자. 
DB에서 데이터를 가져와서 뿌려주고, 일정이 없다면 ``일정 목록이 없습니다.''라는 메시지를 보여준다.
일정 목록을 가져오고나면 목록이 없다는 메시지가 사라지기 때문에, 이 메시지가 있는 상태로 initialLayout을 설정해도 될 것 같은데 사용상의 문제가 있다.\\

단말 전원을 새로 ON시키면 홈 스크린 화면이 보이고 나서 금방 부팅이 완료되는 것이 아니다. 
내부적으로 이후에도 여러 작업들을 한참 하고서야 부팅이 끝났다고 ACTION\_BOOT\_COMPLETED  Action을 발생시키고, 이후에 ACTION\_APPWIDGET\_UPDATE Action을 발생시킨다.\\

화면이 처음 보일 때에 App Widget은 바로 initialLayout 상태 그대로 보여준다. 부팅이 끝나고서야 업데이트를 하므로, 사용자 입장에서는 일정 목록이 없다가 시간이 꽤 지나니까 제대로 보이게 된다.
이런 경우에 사용자는 ``왜 일정이 없지'' 하고서 앱을 실행해 본다. 그리고 다시 홈 스크린으로 돌아오면 일정이 나오기 때문에, 앱을 실행해야만 App Widget이 보이는 것으로 오해하게 된다.\\

이런 경우를 위해서 일반적으로 레이아웃을 만들 때, 다른 부분은 보기 상태를 View.GONE으로 하고, ProgressBar를 포함한 로딩 메시지만을 기본으로 보이도록 한다.

\subsubsection{ICS부터 기본 패딩}
\label{subsubsec:icspadding}
ICS 이전에는 셀 경계까지 꽉 채워서 App Widget을 배치했는데, targetSdkVersion을 14(ICS) 이상으로 하면 App Widget 간 확실한 구분을 위해서 셀 경계와 App Widget 테두리 사이에 기본 패딩이 생긴다. 개발자 가이드에서는 패딩 또는 마진을 혼용해서 용어를 사용하는데 패딩이라고 하는 게 적절할 듯 하다. 마진이라고 하면 셀 경계가 기준이 되어야 하는데 경계는 그냥 표시일 뿐이다. Launcher마다 값은 다르지만 패딩이 조금씩 있는 것을 볼 수 있다.\\

따라서 기존에 꽉 채우는 것을 기준으로 만든 App Widget은 레이아웃을 조정할 필요가 있다. 예를 들어, 일정 4개가 보이기로 한 일정 목록이 targetSdkVersion을 올리면 패딩 때문에 3개 반이 보이는 경우가 생긴다.
%단말 별로 다른 Launcher를 사용하고 있고 패딩값이 모두 다른데 어떻게 계산할까? 바로 AppWidgetHostView의 getDefaultPaddingForWidget 메서드 리턴값을 가지고 사용 가능한 영역을 계산할 수 있다.

\subsubsection{고해상도 단말에서 Bitmap 생성시 메모리 문제}
App Widget의 일반 용도는 단순한 정보를 보여주는 것이다. 여기서 클릭을 하면 Activity를 띄워 상세 정보를 보여주는 식이다.\\

그런데 캘린더 앱/시간표 앱 등 몇 가지 앱은 사용자의 요구로 인해 App Widget의 사이즈가 4x4나 5x5 정도까지 해서 단순 정보가 아닌 화면을 꽉 채운 정보를 전달하는 경우가 있다.
그리고 RemoteViews에서 지원하는 일반적인 뷰 클래스로는 불가능하여 Bitmap을 생성하고, 그 Bitmap에 내용을 그리고서 ImageView에 setBitmap하는 경우가 있다.
게다가 홈 스크린에서는 화면을 크게 차지하면 가급적 바탕화면도 보이도록 투명하게 만들기도 한다.\\

이런 조건들이 있다면 먼저 Bitmap.createBitmap()으로 Bitmap을 생성하고 여기에 그리게 된다.
\begin{verbatim}
Bitmap bitmap = Bitmap.createBitmap(width, height, Bitmap.Config.ARGB_8888); 
\end{verbatim}
createBitmap()의 세 번째 파라미터는 Bitmap.Config API 문서에서 보면, 투명 Bitmap을 생성하기 위해서 ARGB\_8888을 적용한 것이다.
그런데 이 옵션은 픽셀당 4 byte를 차지하고 있다(투명 Alpha 값이 없는 RGB\_565의 2배). 갤노트4의 경우를 보면 해상도가 2560x1440이므로 곱하기를 하면 14,745,600 byte가 나온다. 즉 createBitmap() 만으로도 14M 가량을 사용하게 되는 셈인데, 앱 실행 중에 App Widget Update가 발생한다면 OutOfMemoryError의 원인이 될 수 있다.\\

기존에는 문제가 없던 것이 특히 최신 단말에서 App Widget에서 createBitmap() 실행 중에 OutOfMemoryError로 죽는다면 이것을 원인으로 보면 거의 맞다.
단말 스펙이 높아지면서 해상도가 좋아지는데, 앱 프로세스에서 사용할 수 있는 가용 메모리는 비례해서 커지지 않아서 발생하는 문제이다.
이때는 App Widget을 프로세스 분리하는 방법을 한번 고려해보자.
App Widget 개수가 많다면 그 갯수만큼 프로세스 분리하는 것보다 1개만 분리하는 게 관리가 수월하므로, Service로 App Widget 업데이트 로직을 넘기고 Service를 프로세스 분리해도 된다.
