\begin{comment}
\begin{abstract}
잘 만든 앱은 다 그럭저럭이지만, 그렇지 않은 앱은 제각각의 이유가 있다. 여기서는 그럭저럭한 앱을 만드는 원리와 방법을 찾아보고, 문제를 발생시키는 제각각의 이유에 대해서도 얘기해보자.
\end{abstract}
\end{comment}

\chapter*{이 책의 구성}
안드로이드 컴포넌트에 대한 기본 내용을 중심으로 서술하였다. 각 내용은 근거를 제시하기 위해서 프레임워크 소스나 샘플 소스를 가지고 설명한다.
여러 장에서 반복되는 내용도 있기 때문에 당장 이해되지 않는 부분이 있더라도 끝까지 읽도록 하자.\\

1장에서는 프레임워크 스택의 기본 내용과, 프레임워크 소스를 참고하고 활용하는 방법에 대해 얘기한다.\\

2장에서는 안드로이드 컴포넌트가 실행되는 메인 스레드의 동작 방식을 설명한다.
Handler, Looper, Message, MessageQueue의 관계를 이해하고 나면, 안드로이드 컴포넌트의 여러 실행 문제를 해결할 수 있다. 
개발하면서 골칫거리 가운데 하나인 ANR에 대해서도 원인과 결과, 그리고 해결 방식에 대해서 다룬다.\\

3장에서는 2장의 Handler와 내용이 연결되는 HandlerThread와 스레드 풀, AsyncTask에 대한 내용을 다룬다.\\

4장에서는 Activity, Service, Application의 상위 클래스이면서, 안드로이드 컴포넌트를 실행하거나 리소스를 참조할 때 필요한 Context 클래스에 대해서 살펴본다.\\

5장부터 9장까지는 Activity, Service, ContentProvider, BroadcastReceiver, Application까지 이슈 중심으로 안드로이드 컴포넌트를 설명한다.\\

10장에서는 시스템 서비스 목록을 정리하고, 시스템 서비스와 Service 컴포넌트와 차이점을 얘기한다. 시스템의 상태를 알기 위해 dumpsys 명령어를 활용하는 방법과 시스템 서비스의 여러 이슈에 대해서도 살펴본다.\\

11장에서는 앱 개발에서 사용하는 구현 패턴을 얘기한다. 싱글톤과 마커 인터페이스, Fragment 정적 생성 항목을 언급한다.\\

프레임워크 소스와 샘플은 안드로이드 버전에 관계 없이 공통되는 내용으로 주로 설명하였는데, 버전에 따라 동작이 바뀌는 게 있다면 가급적 언급하였다. 
언급한 내용 외에도 버전에 따른 내용이 달라서 혼동이 되는 부분이 있다면 이메일(suribada@gmail.com)로 알려주기 바란다.

\addcontentsline{toc}{chapter}{이 책의 구성}