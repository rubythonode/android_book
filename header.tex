\chapter*{머리말}
\section*{대상 독자}
이 책은 안드로이드 앱을 만들어본 경험을 가진 개발자를 대상으로 한 것이다. 
앱을 개발하면서 자신이 정말 제대로 만들고 있는 지, 문제를 올바르게 해결하고 있는 지 의문을 가지고 있었다면 이 책이 도움이 될 것이다.\\

이 책에서는 기본 원리를 이해하고 이를 올바르게 적용하는 것에 중점을 두었다. 
다른 책에서는 작게 다뤄지는 내용들이라도 실무에서 중요하다고 생각하는 것에 많은 내용을 할애하였다.
기초 서적을 보고 나서는 그 다음에 볼만한 게 많지 않은데 이것도 `두 번째 책' 가운데 하나가 되었으면 하는 바램이다.\\

경험자를 대상으로 하므로 기초적인 내용은 안다는 전제에서 시작하는 내용이 많다. 
전혀 생소한 내용이라면 관련해서 검색을 통해 찾아보고, 분량이 많지도 않으므로 어쨌든 쭉 읽어보고 반복해서 읽는 것을 추천한다. 
내용 가운데서 독자가 정말 궁금하고 알고 싶던 것이 있다면 그건 필자의 기쁨이 될 것이다.

\section*{에피소드}
필자는 5년이 넘는 기간 동안 다양한 앱을 겪어왔다. 물론 그 가운데서 잘 만든 것도 있지만 그렇지 않은 것도 많았다. 
안드로이드의 기본 원리를 알고 있다면, 해서는 안되는 수많은 코드들이 있었다.
남들이 만든 코드도 그렇지만 필자가 만든 것도 나중에 생각해보면 왜 그랬을까 하는 부끄러운 것도 없지 않다.\\

안드로이드가 문서화가 잘 되어 있는 편이 아니다 보니 기본 원리를 알려주는 내용이 많지 않다. 
기본 원리를 모르는 상태에서 문제가 없어 보이는 정도의 수준에서만 개발하다 보면, 군더더기 코드를 양산하고 요구 사항이나 안드로이드 버전, 다양한 단말 같은 환경 변화에 취약해지는 문제가 생긴다.
문제가 발생하면 어떻게든 꼼수로 해결하려는 많은 시도를 봐왔는데, 결국 문제 해결은 먼 곳이 아니라 기본 원리에서부터 시작해야만
단순하고 확실하게 해결할 수 있다는 것을 경험으로 알게 되었다.\\

기본 원리를 알려면 결국 안드로이드 내부 구조에 대한 이해가 필요하다.
내부 구조에 대해서 설명하는 어려운 책이나 강의도 있지만, 
그 내용이 어려운 수준에서 끝나고 실무에 어떻게 잘 적용하면 되는지는 또 각자의 몫이 되었다. 
이론과 실무의 차이를 좁히려는 시도로서 스터디를 꾸리고 강의를 하면서 
남에게 전달하기 위해서 더 깊이 있게 탐구하게 되었다.
그리고 결과물로서 책을 내는 시도를 하게 되었는데, 단 한 줄의 내용을 쓰기 위해 며칠을 테스트하고 고민하기도 하면서 분량에 비해 많은 시간이 소요되었다.
공부하면 할수록 공부할 게 많아지는 건 한동안 나의 일상이 될 것 같다.

\section*{감사의 글}
내용 검토를 해준 이효근 님, 윤신주 님, 김태중 님, 김성수 님, 원형식 님, 송지철 님, 이정민 님, 임원석 님, 이종권 님에게 감사드린다. 
이 분들 덕분에 내용에서 많은 문제들이 수정될 수 있었다. 이 책은 혼자서 한 게 아니라 함께 만들어낸 것이다.\\

내가 IT 업계에서 밥을 먹고 살 수 있도록 많은 도움과 격려를 해준 이창신 군, 최희탁 군, 강용석 군에게도 고마움을 전해야겠다.
오래 전 잠시 IT 업계를 떠나있을 때 동고동락한 이정훈 형도 내게 고마운 사람이다.\\

항상 내 편이 되어준 부모님과 형제들, 
오랜 시간을 함께하고 기다려준 아내에게도 감사한다. 삶의 이유가 되어주는 현종과 서윤에게도 사랑과 고마움을 전한다.

\begin{comment}
필자는 회사에서 능력을 인정받거나 화려한 기술을 가진 것도 아닌 평범한 개발자이다. 
다만 문제를 겪고 해결할 때마다 메모하고 시간을 내서 내용을 정리한 것 뿐이다.
정리한 내용은 지력이 갈수록 떨어지는 필자를 위한 것이기도 하다.\\

나도 책과 인터넷에서 많은 도움을 받았다. 
오래 전 팀 동료가 그런 얘기를 했다. ``국내 IT서적은 다 쓰레기다'' 그 동료는 책 꽂이에 원서만 나열해놓고 있었다.
어쩌면 쓰레기 하나를 더 만들었는지도 모르겠다.

좌충우돌한 이야기

예민한 내용이 많아서 오류가 있을 수 있다. 혹시 오류라고 생각한다면 제보 바란다.

죄와벌 이나 태백산맥 같은 역작을 내는 것도 아닌데, 무슨 시간이 이리 많이 걸리는지..
\end{comment}